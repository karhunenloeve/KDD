\ifx\pdfminorversion\undefined\else\pdfminorversion=4\fi
\documentclass[aspectratio=169,t]{beamer}
%\documentclass[aspectratio=169,t,handout]{beamer}

% English version FAU Logo
\usepackage[english]{babel}
% German version FAU Logo
%\usepackage[ngerman]{babel}

\usepackage[utf8]{inputenc}
\usepackage[T1]{fontenc}
\usepackage{amsmath,amssymb}
\usepackage{graphicx}
\usepackage{listings}
\usepackage{url}
\usepackage{enumitem}
\usepackage{hyperref}
\usepackage{fontawesome}
\usepackage{graphicx}
\usepackage{booktabs}
\usepackage{calc}
\usepackage{ifthen}
\usepackage{xcolor}
\usepackage{tabularx}
\usepackage{tikz}
\usepackage{tikz}
\usepackage{tikz-cd}
\usepackage{verbatim}
\usepackage{pgfplots,pgfplotstable,pgf-pie}
\usepackage{filecontents}
\newcommand{\plots}{0.611201}
\newcommand{\plotm}{2.19882}
\pgfplotsset{height=4cm,width=8cm,compat=1.17}
\pgfmathdeclarefunction{gauss}{2}{%
  \pgfmathparse{1/(#2*sqrt(2*pi))*exp(-((x-#1)^2)/(2*#2^2))}%
}

\tikzset{
    vertex/.style = {
        circle,
        fill            = black,
        outer sep = 2pt,
        inner sep = 1pt,
    }
}

\tikzset{
    mynode/.style={
        draw,
        thick,
        anchor=south west,
        minimum width=2cm,
        minimum height=1.3cm,
        align=center, 
        inner sep=0.2cm,
        outer sep=0,
        rectangle split, 
        rectangle split parts=2,
        rectangle split draw splits=false},
    reverseclip/.style={
        insert path={(current page.north east) --
            (current page.south east) --
            (current page.south west) --
            (current page.north west) --
            (current page.north east)}
    }
}

\tikzset{basic/.style={
        draw,
        rectangle split,
        rectangle split parts=2,
        rectangle split part fill={blue!20,white},
        minimum width=2.5cm,
        text width=2cm,
        align=left,
        font=\itshape
    },
    Diamond/.style={ diamond, 
                      draw, 
                      shape aspect=2, 
                      inner sep = 2pt,
                      text centered,
                      fill=blue!10!white,
                      font=\itshape
                    }}


\tikzset{level 1/.append style={sibling angle=50,level distance = 165mm}}
\tikzset{level 2/.append style={sibling angle=20,level distance = 45mm}}
\tikzset{every node/.append style={scale=1}}

\usetikzlibrary{arrows,decorations.pathmorphing,backgrounds,fit,positioning,shapes.symbols,chains,intersections,snakes,positioning,matrix,mindmap,shapes.multipart,shapes,calc,shapes.geometric}

% read in data file


\newcommand{\MaxNumberX}{3}
\newcommand{\MaxNumberY}{5}
\newcommand{\tikzmark}[1]{\tikz[remember picture] \node[coordinate] (#1) {#1};}

\pgfplotstableread{data/iris.dat}\iris
\pgfplotstablegetrowsof{\iris}
\pgfplotsset{compat=1.14}
\pgfmathsetmacro\NumRows{\pgfplotsretval-1}
\definecolor{airforceblue}{rgb}{0.36, 0.54, 0.66}

\usepgfplotslibrary{groupplots}
% Options:
%  - inst:      Institute
%                 med:      MedFak FAU theme
%                 nat:      NatFak FAU theme
%                 phil:     PhilFak FAU theme
%                 rw:       RWFak FAU theme
%                 rw-jura:  RWFak FB Jura FAU theme
%                 rw-wiso:  RWFak FB WISO FAU theme
%                 tf:       TechFak FAU theme
%  - image:     Cover image on title page
%  - plain:     Plain title page
%  - longtitle: Title page layout for long title
\usetheme[%
  image,%
  longtitle,%
  tf
]{fau}

% Enable semi-transparent animation preview
\setbeamercovered{transparent}


\lstset{%
  language=Python,
  tabsize=2,
  basicstyle=\tt,
  keywordstyle=\color{blue},
  commentstyle=\color{green!50!black},
  stringstyle=\color{red},
  numbers=left,
  numbersep=0.5em,
  xleftmargin=1em,
  numberstyle=\tt
}


% Title, authors, and date
\title[KDD]{Chapter IV: OLAP}
\subtitle{Knowledge Discovery in Databases}
\author[L.~Melodia]{Luciano Melodia M.A.}
% English version
\institute[Department]{Evolutionary Data Management, Friedrich-Alexander University Erlangen-Nürnberg}
% German version
%\institute[Lehrstuhl]{Lehrstuhl, Friedrich-Alexander-Universit\"at Erlangen-N\"urnberg}
\date{Summer semester 2021}
% Set additional logo (overwrites FAU seal)
%\logo{\includegraphics[width=.15\textwidth]{themefau/art/xxx/xxx.pdf}}
\begin{document}
  % Title
  \maketitle

  { 
    \setbeamertemplate{footline}{}
    \begin{frame}{Chapter IV: Data warehousing and online analytical processing}
        \begin{itemize}
            \item \textbf{Data warehouse: basic concepts.}
            \item Data warehouse modeling: data cube and OLAP.
            \item Data warehouse design and usage.
            \item Data warehouse Implementation.
            \item Data generalization by attribute-oriented induction.
            \item Summary.
        \end{itemize}
    \end{frame}
  }

  {
    \setbeamertemplate{footline}{}
    \begin{frame}{References (I)}
    \begin{itemize}
      \item S. Agarwal, R. Agrawal, P. M. Deshpande, A. Gupta, J. F. Naughton, R. Ramakrishnan, and S. Sarawagi: On the computation of multidimensional aggregates. VLDB'96.
      \item D. Agrawal, A. E. Abbadi, A. Singh, and T. Yurek: Efficient view maintenance in data warehouses. SIGMOD'97.
      \item R. Agrawal, A. Gupta, and S. Sarawagi: Modeling multidimensional databases.  ICDE'97.
      \item {\color{airforceblue}S. Chaudhuri and U. Dayal: An overview of data warehousing and OLAP technology. ACM SIGMOD Record, 26:65-74, 1997-}
      \item E. F. Codd, S. B. Codd, and C. T. Salley: Beyond decision support. Computer World, 27, July 1993.
      \item J. Gray, et al. Data cube: A relational aggregation operator generalizing group-by, cross-tab and sub-totals. Data Mining and Knowledge Discovery, 1:29-54, 1997.
      \item A. Gupta and I. S. Mumick.  Materialized Views: Techniques, Implementations, and Applications. MIT Press, 1999.
      \item J. Han:  Towards on-line analytical mining in large databases. ACM SIGMOD Record, 27:97-107, 1998.
    \end{itemize}
    \end{frame}
  }

  {
    \setbeamertemplate{footline}{}
    \begin{frame}{References (II)}
    \begin{itemize}
      \item V. Harinarayan, A. Rajaraman, and J. D. Ullman: Implementing data cubes efficiently. SIGMOD'96.
      \item J. Hellerstein, P. Haas, and H. Wang: Online aggregation. SIGMOD'97.
      \item C. Imhoff, N. Galemmo, and J. G. Geiger: Mastering Data Warehouse Design: Relational and Dimensional Techniques. John Wiley, 2003.
      \item W. H. Inmon: Building the Data Warehouse. John Wiley, 1996.
      \item R. Kimball and M. Ross: The Data Warehouse Toolkit: The Complete Guide to Dimensional Modeling. 2ed. John Wiley, 2002.
      \item P. O’Neil and G. Graefe: Multi-table joins through bitmapped join indices. ACM SIGMOD Record, 24:8–11, Sept. 1995.
      \item P. O'Neil and D. Quass: Improved query performance with variant indexes. SIGMOD'97.
      \item Microsoft. OLEDB for OLAP programmer's reference version 1.0. In \href{http://www.microsoft.com/data/oledb/olap}{http://www.microsoft.com/data/oledb/olap}, 1998.
    \end{itemize}
    \end{frame}
  }

  { % Questions?
    \setbeamertemplate{footline}{}
    \begin{frame}[c]
      \begin{center}
        Thank you for your attention.\\
        {\bf Any questions about the fourth chapter?}\\[0.5cm]
        Ask them now, or again, drop me a line: \\ 
        \faSendO \ \texttt{luciano.melodia@fau.de}.
      \end{center}
    \end{frame}
  }
\end{document}

